\documentclass{article}
\usepackage[letterpaper, margin=1in]{geometry}
%\usepackage{graphicx}
%\usepackage{caption}
%\usepackage{cancel}
%\usepackage{amsmath}
%\usepackage{mathrsfs}
%\usepackage{ulem}
%\usepackage[version=4]{mhchem}
\usepackage{comment}
%\usepackage{float}
%\usepackage{enumitem}
\usepackage{soul}
\usepackage{float}
\usepackage{siunitx}
%\usepackage{tabularx}
%\usepackage{bm}
\usepackage{multicol}
%\usepackage{mathtools}
\usepackage{fontspec}
%\usepackage{pgfplots}
%\usepackage{pgfplotstable}
%\usepgfplotslibrary{statistics}
%\pgfplotsset{compat=1.18}
\setmainfont{Times New Roman}
\usepackage{setspace}
\onehalfspacing
\usepackage{biblatex} %[backend=biber]
\addbibresource{main.bib}

%cSpell:ignore Lombardo Brodee NUCE Geuther Ortec Canberra Tektronix Trifluoride Thermalization Epithermal LINEST

\begin{document}

\centering{ \section*{Optimization for Lithium Breeding in the Blanket and Shielding for the STAR Power Plant}

%Andrei Khodak$^1$,% 
Rocco Lombardo$^1$, Sean Dailey$^2$, Vasil Ivakimov$^3$, Issac James$^4$ 

\bigskip

%$^1$Author Affiliation: Princeton Plasma Physics Laboratory, Princeton, NJ 

$^{1,2,3,4}$Author Affiliation: Pennsylvania State University, State College, PA

\bigskip

With the intention for the spherical tokamak advanced reactor (STAR) to generate energy; It uses a D-T reaction to generate high-energy neutrons. 
These D-T reactions have a major advantage over other fusion reactions in the fact that they have highest cross section of occurrence, compared to other popular fusion reactions, as well as the highest power density, at $\SI{34}{\watt\per\meter\cubed\kilo\pascal\squared}$
\cite{Khodak2025}. 
However, there are multiple issues associated with the use of tritium in this reaction. The first is that tritium has a relatively short half-life of 2 years,
which makes it difficult to accumulate and store for this reaction. Second, the energy emitted in this reaction is mostly transferred to the emitted neutron,
leading to extremely high energy neutrons, 14 MeV. To resolve these issues, lithium-6 is used with a neutron source to produce both tritium and a small
number of neutrons. This occurs within the blanket, which contains the lithium-6 to breed tritium. Once the D-T reaction occurs, the kinetic energy of the 14
MeV neutrons must be converted to a more useful energy form. The current moderator of choice, within the blanket, is lead.
The primary focus of our work is optimizing the blanket's configuration to promote tritium production for this D-T reaction. The secondary focus is the
optimization of the neutron shield to protect other components and personnel.

\bigskip

%Format the document with 1-inch margins at the top, bottom, left, and right. The abstract must fit on one page. Up to one figure or table is allowed.  Boldfaced font style is used only for the paper title. Use one line of space between the title, author(s), affiliation(s), and the abstract text.  
}
\printbibliography

\end{document}
