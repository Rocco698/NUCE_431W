\documentclass{article}
\usepackage[letterpaper, margin=1in]{geometry}
%\usepackage{graphicx}
%\usepackage{caption}
%\usepackage{cancel}
%\usepackage{amsmath}
%\usepackage{mathrsfs}
%\usepackage{ulem}
%\usepackage[version=4]{mhchem}
\usepackage{comment}
%\usepackage{float}
%\usepackage{enumitem}
\usepackage{soul}
\usepackage{float}
%\usepackage{siunitx}
%\usepackage{tabularx}
%\usepackage{bm}
\usepackage{multicol}
%\usepackage{mathtools}
\usepackage{fontspec}
%\usepackage{pgfplots}
%\usepackage{pgfplotstable}
%\usepgfplotslibrary{statistics}
%\pgfplotsset{compat=1.18}
\setmainfont{Times New Roman}

%cSpell:ignore Lombardo Brodee NUCE Geuther Ortec Canberra Tektronix Trifluoride Thermalization Epithermal LINEST

\begin{document}

\centering{ \section*{Lithium Blanket and Shielding Design for STAR Power Plant}

Andrei Khodak$^1$, Rocco Lombardo$^2$, Sean Dailey$^3$, Vasil Ivakimov$^4$, Issac James$^5$, 

\bigskip

$^1$Author Affiliation: Princeton Plasma Physics Laboratory, Princeton, NJ 

$^{2,3,4,5}$Author Affiliation: Pennsylvania State University, State College, PA

\bigskip

The intention for fusion power is to generate energy. To this end, the spherical tokamak advanced reactor (STAR) uses a D-T reaction to generate high-energy neutrons. 
D-T reactions have a major advantage over other fusion reactions in the fact that they have the highest probability of occurrence as well as the highest power density, at 34 W/m^3/kPa^2. However, multiple issues arise with the use of Tiritum in this reaction. The first issue is the difficulty in obtaining Tritium, as it is not produced naturally. Second, tritium has a relatively short half-life of 12 years, which makes it difficult to accumulate and store for this reaction. Third, the energy emitted in this reaction is mostly transferred to the neutron products, leading to extermely high energy neutrons. To resolve these issues, lithium-6 is used with a neutron source to produce both tritium and a low number of neutrons. This occurs within the blanket, which contains the lithium-6 to breed tritium. Once the D-T reaction occurs, the kinetic energy of the 14 MeV neutrons must be converted to a more useful energy form. Currently, the method of choice is to moderate neutrons with lead.

\bigskip

The primary focus of our work is optimizing the blanket's configuration to promote tritium production for the D-T reaction. A secondary focus is the development of the 
neutron shield to protect other components and personnel.

\bigskip

%Format the document with 1-inch margins at the top, bottom, left, and right. The abstract must fit on one page. Up to one figure or table is allowed.  Boldfaced font style is used only for the paper title. Use one line of space between the title, author(s), affiliation(s), and the abstract text.  
}


\end{document}
